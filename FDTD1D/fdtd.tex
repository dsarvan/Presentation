\documentclass[8pt]{beamer}
\mode<presentation>{}

\usepackage{times}
\usepackage{amsmath}
\usepackage{amsfonts}
\usepackage{amssymb}
\usepackage{physics}
\usepackage{graphicx}
\usepackage{booktabs}
\usepackage{xspace}
\usepackage[font=small,skip=0pt]{caption}

\graphicspath{{/home/saran/Presentation/FDTD1D/Programs/}}
\usetheme{metropolis}
\setbeamercolor{frametitle}{fg=mDarkTeal, bg=black!2}
\usecolortheme[snowy]{owl}

\title{Finite-Difference Time-Domain Method}
\author{\normalsize{D. Saravanan}}
%\institute{
%    \normalsize{Department of Physics}\\
%    \normalsize{Indian Institute of Technology Madras}
%    }
\date{\today}

\begin{document}

% frame 1
\begin{frame}
\titlepage
\end{frame}

% frame 2
\begin{frame}
\frametitle{Introduction}
To learn and do three-dimensional electromagnetic simulation using the finite-difference
time-domain (FDTD) method.

Type of material:
\begin{enumerate}
\item Free space
\item Complex dielectric material
\item Frequency-dependent material
\end{enumerate}
\end{frame}

% frame 3
\begin{frame}[allowframebreaks]{Outline}
\frametitle{Formulation}
Some choice that have been made:
\begin{enumerate}
\item The use of Normalised Units: \\
Maxwell's equations have been normalized by substituting
\begin{equation*}
\widetilde{E} = \displaystyle \sqrt{\frac{\varepsilon_{0}}{\mu_{0}}} E
\end{equation*}

this is a system similar to Gaussian units.

The reason for using it here is the simplicity in the formulation. The $E$ and $H$ fields
have the same order of magnitude. This has an advantage in formulating the PML.

\item Maxwell's Equations with the Flux Density: \\
Time-domain Maxwell's equations from which the FDTD formulation is developed. \\

straight forward formulation:

\begin{equation*}
\frac{\partial E}{\partial t} = \frac{1}{\varepsilon_{0}} \curl{H}
\end{equation*}

\begin{equation*}
\frac{\partial H}{\partial t} = - \frac{1}{\mu_{0}} \curl{E}
\end{equation*}

\vspace{1cm}

formulation using the flux density:

\begin{equation*}
\frac{\partial D}{\partial t} = \curl{H}
\end{equation*}

\begin{equation*}
D = \varepsilon_{0} \varepsilon^{*}_{r} E
\end{equation*}

\begin{equation*}
\frac{\partial H}{\partial t} = - \frac{1}{\mu_{0}} \curl{E}
\end{equation*}

in this formulation, it is assumed that the materials
being simulated are non-magnetic, that is, $H = (1/\mu_{0}) B$
\end{enumerate}
\end{frame}

% frame 4
\begin{frame}
\frametitle{Pulse propagating in free space in one-dimension}
Time-dependent Maxwell's curl equations for free space:

\begin{equation*}
\frac{\partial E}{\partial t} = \frac{1}{\varepsilon_{0}} \curl{H}
\end{equation*}

\begin{equation*}
\frac{\partial H}{\partial t} = - \frac{1}{\mu_{0}} \curl{E}
\end{equation*}

simple one-dimensional case:
\begin{equation*}
\frac{\partial E_{x}}{\partial t} = - \frac{1}{\varepsilon_{0}} \frac{\partial H_{y}}{\partial z}
\end{equation*}

\begin{equation*}
\frac{\partial H_{y}}{\partial t} = - \frac{1}{\mu_{0}} \frac{\partial E_{x}}{\partial z}
\end{equation*}
\end{frame}

% frame 5
\begin{frame}
\frametitle{Pulse propagating in free space in one-dimension}
\begin{itemize}
\item The formulation of Equations assume that the $E$ and $H$ fields are interleaved in both
space and time.
\item The new value of $E_{x}$ is calculated from the previous value of $E_{x}$ and the
most resent values of $H_{y}$. This is the fundamental paradigm of the FDTD method.
\end{itemize}

governing equations,

\begin{equation*}
\widetilde{E}_{x}^{n + 1/2} (k) = \widetilde{E}_{x}^{n - 1/2} (k) - \frac{\Delta
t}{\sqrt{\varepsilon_{0} \mu_{0}} \cdot \Delta x} \Bigg[H_{y}^{n} \left(k + \frac{1}{2}\right)
- H_{y}^{n} \left(k - \frac{1}{2}\right)\Bigg]
\end{equation*}

\begin{equation*}
H_{y}^{n + 1} \left(k + \frac{1}{2}\right) = H_{y}^{n} \left(k + \frac{1}{2}\right) -
\frac{\Delta t}{\sqrt{\varepsilon_{0} \mu_{0}} \cdot \Delta x}  \Bigg[\widetilde{E}_{x}^{n + 1/2} (k + 1) 
- \widetilde{E}_{x}^{n + 1/2} (k)\Bigg]
\end{equation*}
\end{frame}

% frame 6
\begin{frame}
\frametitle{Pulse propagating in free space in one-dimension}
Once the cell size $\Delta x$ is choosen, then the time step $\Delta t$ is determined by

\begin{equation*}
\Delta t = \frac{\Delta x}{2 \cdot c_{0}},
\end{equation*}

where $c_{0}$ is the speed of light in free space. Therefore, remembering that
$\varepsilon_{0} \mu_{0} = 1/(c_{0})^{2}$,

\begin{equation*}
\frac{\Delta t}{\sqrt{\varepsilon_{0} \mu_{0}} \cdot \Delta x} = \frac{\Delta x}{2 \cdot
c_{0}} \cdot \frac{1}{\sqrt{\varepsilon_{0} \mu_{0}} \cdot \Delta x} = \frac{1}{2}
\end{equation*}

\begin{equation*}
\widetilde{E}_{x}^{n + 1/2} (k) = \widetilde{E}_{x}^{n - 1/2} (k) - \frac{1}{2} \Bigg[H_{y}^{n}
\left(k + \frac{1}{2}\right) - H_{y}^{n} \left(k - \frac{1}{2}\right)\Bigg]
\end{equation*}

\begin{equation*}
H_{y}^{n + 1} \left(k + \frac{1}{2}\right) = H_{y}^{n} \left(k + \frac{1}{2}\right) -
\frac{1}{2} \Bigg[\widetilde{E}_{x}^{n + 1/2} (k + 1) - \widetilde{E}_{x}^{n + 1/2}
(k)\Bigg]
\end{equation*}
\end{frame}

% frame 7
\begin{frame}
\frametitle{Simulation in free space}
\begin{figure}[ht!]
\includegraphics[width=0.9\linewidth]{Problem_1_1/fd1d_1_1.pdf}
\caption{FDTD simulation of a pulse in free space after 100 time steps. The pulse
originated in the center and travels outward.}
\end{figure}
\end{frame}

% frame 8
\begin{frame}
\frametitle{Simulation in free space}
\begin{figure}[ht!]
\includegraphics[width=0.9\linewidth]{Problem_1_1/fd1d_1_2.pdf}
\caption{FDTD simulation of a pulse in free space after 100 time steps. It has two
sources, one at $kc - 20$ and one at $kc + 20$}
\end{figure}
\end{frame}

% frame 9
\begin{frame}
\frametitle{Simulation in free space}
\begin{figure}[ht!]
\includegraphics[width=0.9\linewidth]{Problem_1_1/fd1d_1_3.pdf}
\caption{FDTD simulation of a pulse in free space after 100 time steps. Instead of $E_{x}$
as the source, use $H_{y}$ at $k = kc$ as the source}
\end{figure}
\end{frame}

% frame 10
\begin{frame}
\frametitle{Simulation in free space}
\begin{figure}[ht!]
\includegraphics[width=0.9\linewidth]{Problem_1_1/fd1d_1_4.pdf}
\caption{FDTD simulation of a pulse in free space after 100 time steps. Instead of $E_{x}$
as the source, use a two-point magnetic source at $kc - 1$ and $kc$ such that $hy[kc - 1]
= -hy[kc]$}
\end{figure}
\end{frame}

% frame 11
\begin{frame}
\frametitle{Stability and the FDTD method}
\begin{itemize}
\item An EM wave propagating in free space cannot go faster than the speed of light.
\item To propagate a distance of one cell requires a minimum time of $\Delta t = \Delta
x/c_{0}$.
\item With a two-dimensional simulation, we must allow for the propagation in the diagonal
direction, which brings the requirement to $\Delta t = \Delta x/(\sqrt{2} c_{0})$.
\item With a three-dimensional simulation requires $\Delta t = \Delta x/(\sqrt{3} c_{0})$.
\item This is summarized by the Courant Condition
\begin{equation*}
\Delta t = \displaystyle \frac{\Delta x}{\sqrt{n} \cdot c_{0}},
\end{equation*}
where $n$ is the dimension of the simulation.
\end{itemize}
\end{frame}

% frame 12
\begin{frame}
\frametitle{Absorbing boundary condition in one dimension}
\begin{itemize}
\item Absorbing boundary conditions are necessary to keep outgoing $E$ and $H$ fields from
being reflected back into the problem space.
\item If a wave is going toward a boundary in free space, it is traveling at $c_{0}$, the
speed of light.
\item In one time step of the FDTD algorithem, it travels
\begin{equation*}
Distance = c_{0} \cdot \Delta t = c_{0} \cdot \frac{\Delta x}{2 \cdot c_{0}} =
\frac{\Delta x}{2}
\end{equation*}
\item It takes two time steps for the field to cross one cell. Thus an acceptable boundary
condition might be
\begin{equation*}
E_{x}^{n} (0) = E_{x}^{n-2} (1)
\end{equation*}
\end{itemize}
\end{frame}

% frame 13
\begin{frame}
\frametitle{Absorbing boundary condition in one dimension}
\begin{figure}[ht!]
\includegraphics[width=0.9\linewidth]{Problem_1_3/fd1d_1_1.pdf}
\caption{Simulation of an FDTD program with absorbing boundary conditions. Notice that the
pulse is absorbed at the edges without reflecting anything back.}
\end{figure}
\end{frame}

% frame 14
\begin{frame}
\frametitle{Propagation in a Dielectric medium}
To simulate a medium with a dielectric constant other than $1$, we have to add the
relative dielectric constant $\varepsilon_{r}$ to Maxwell's equations:

\begin{equation*}
\frac{\partial E}{\partial t} = \frac{1}{\varepsilon_{0} \varepsilon_{r}} \curl{H}
\end{equation*}

\begin{equation*}
\frac{\partial H}{\partial t} = - \frac{1}{\mu_{0}} \curl{E}
\end{equation*}

simple one-dimensional case:
\begin{equation*}
\frac{\partial E_{x}}{\partial t} = - \frac{1}{\varepsilon_{0} \varepsilon_{r}} \frac{\partial H_{y}}{\partial z}
\end{equation*}

\begin{equation*}
\frac{\partial H_{y}}{\partial t} = - \frac{1}{\mu_{0}} \frac{\partial E_{x}}{\partial z}
\end{equation*}
\end{frame}

% frame 15
\begin{frame}
\frametitle{Propagation in a Dielectric medium}
governing equations,

\begin{equation*}
\widetilde{E}_{x}^{n + 1/2} (k) = \widetilde{E}_{x}^{n - 1/2} (k) - \frac{1}{2 \cdot
\varepsilon_{r}} \Bigg[H_{y}^{n}
\left(k + \frac{1}{2}\right) - H_{y}^{n} \left(k - \frac{1}{2}\right)\Bigg]
\end{equation*}

\begin{equation*}
H_{y}^{n + 1} \left(k + \frac{1}{2}\right) = H_{y}^{n} \left(k + \frac{1}{2}\right) -
\frac{1}{2} \Bigg[\widetilde{E}_{x}^{n + 1/2} (k + 1) - \widetilde{E}_{x}^{n + 1/2}
(k)\Bigg]
\end{equation*}
\end{frame}

% frame 16
\begin{frame}
\frametitle{Propagation in a Dielectric medium}
\begin{figure}[ht!]
\includegraphics[width=1.0\linewidth]{Problem_1_4/fd1d_1_1.pdf}
\caption{Simulation of a pulse striking dielectric material with a dielectric constant of
$4$. The source originates at cell number $5$.}
\end{figure}
\end{frame}

% frame 17
\begin{frame}
\frametitle{Simulating with a sinusoidal source}
\begin{figure}[ht!]
\includegraphics[width=1.0\linewidth]{Problem_1_5/fd1d_1_1.pdf}
\caption{Simulation of a propagating sinusoidal wave of $700$ MHz striking a medium with a
relative dielectric constant of $\varepsilon_{r} = 4$.}
\end{figure}
\end{frame}

% frame 18
\begin{frame}
\frametitle{Simulating with a sinusoidal source}
\begin{figure}[ht!]
\includegraphics[width=1.0\linewidth]{Problem_1_6/fd1d_1_1.pdf}
\caption{Simulation of a propagating sinusoidal wave of $3$ GHz striking a medium with a
relative dielectric constant of $\varepsilon_{r} = 20$.}
\end{figure}
\end{frame}

% frame 19
\begin{frame}
\frametitle{Propagation in a lossy dielectric medium}
general form of time-dependent Maxwell's curl equations,

\begin{equation*}
\varepsilon_{r} \varepsilon_{0} \frac{\partial E}{\partial t} = \curl{H} - J
\end{equation*}

\begin{equation*}
\frac{\partial H}{\partial t} = - \frac{1}{\mu_{0}} \curl{E}
\end{equation*}

$J$, the current density, can also be written as
\begin{equation*}
J = \sigma E
\end{equation*}
where $\sigma$ is the conductivity. 

substituting,
\begin{equation*}
\frac{\partial E}{\partial t} = \frac{1}{\varepsilon_{r} \varepsilon_{0}} \curl{H} -
\frac{\sigma}{\varepsilon_{r} \varepsilon_{0}} E
\end{equation*}

simple one-dimensional equation:
\begin{equation*}
\frac{\partial E_{x} (t)}{\partial t} = - \frac{1}{\varepsilon_{r} \varepsilon_{0}}
\frac{\partial H_{y} (t)}{\partial z} - \frac{\sigma}{\varepsilon_{r} \varepsilon_{0}}
E_{x} (t)
\end{equation*}
\end{frame}

% frame 20
\begin{frame}
\frametitle{Propagation in a lossy dielectric medium}
using the change of variables,
\begin{equation*}
\frac{\partial \widetilde{E}_{x} (t)}{\partial t} = - \frac{1}{\varepsilon_{r}
\sqrt{\mu_{0} \varepsilon_{0}}} \frac{\partial H_{y} (t)}{\partial z} -
\frac{\sigma}{\varepsilon_{r} \varepsilon_{0}} \widetilde{E}_{x} (t)
\end{equation*}

\begin{equation*}
\frac{\partial H_{y} (t)}{\partial t} = - \frac{1}{\sqrt{\mu_{0} \varepsilon_{0}}}
\frac{\partial \widetilde{E}_{x} (t)}{\partial t}
\end{equation*}

governing equations,
\begin{equation*}
\widetilde{E}_{x}^{n + 1/2} (k) = \frac{\displaystyle \left(1 - \frac{\Delta t \cdot \sigma}{2
\varepsilon_{r} \varepsilon_{0}}\right)}{\displaystyle \left(1 + \frac{\Delta t \cdot \sigma}{2 \varepsilon_{r}
\varepsilon_{0}}\right)} \widetilde{E}_{x}^{n - 1/2} (k) - \frac{\displaystyle
\left(\frac{1}{2}\right)}{\displaystyle \varepsilon_{r} \left(1 + \frac{\Delta t \cdot
\sigma}{2 \varepsilon_{r} \varepsilon_{0}}\right)} \Bigg[H_{y}^{n} \left(k +
\frac{1}{2}\right) - H_{y}^{n} \left(k - \frac{1}{2}\right)\Bigg]
\end{equation*}

\begin{equation*}
H_{y}^{n + 1} \left(k + \frac{1}{2}\right) = H_{y}^{n} \left(k + \frac{1}{2}\right) -
\frac{1}{2}  \Bigg[\widetilde{E}_{x}^{n + 1/2} (k + 1) - \widetilde{E}_{x}^{n + 1/2} (k)\Bigg]
\end{equation*}
\end{frame}

\begin{frame}
\frametitle{Propagation in a lossy dielectric medium}
\begin{figure}[ht!]
\includegraphics[width=0.9\linewidth]{Problem_1_7/fd1d_1_1.pdf}
\caption{Simulation of a propagating sinusoidal wave striking a lossy dielectric material
with a dielectric constant of $4$ and a conductivity of $0.04$ (S/m). The source is $700$
MHz and originates at cell number $5$.}
\end{figure}
\end{frame}

\end{document}

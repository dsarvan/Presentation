\documentclass[8pt]{beamer}
\mode<presentation>{}

\usepackage{times}
\usepackage{amsmath}
\usepackage{amsfonts}
\usepackage{amssymb}
\usepackage{physics}
\usepackage{graphicx}
\usepackage{booktabs}
\usepackage{xspace}

\graphicspath{{/home/saran/Presentation/FDTD1D/Programs/Images/}}
\usetheme{metropolis}
\setbeamercolor{frametitle}{fg=mDarkTeal, bg=black!2}
\usecolortheme[snowy]{owl}

\title{Finite Difference Time Domain}
\author{\normalsize{D. Saravanan}}
\institute{
    \normalsize{Department of Physics}\\
    \normalsize{Indian Institute of Technology Madras}
    }
\date{\today}

\begin{document}

\begin{frame}
\titlepage
\end{frame}

\begin{frame}
\frametitle{Introduction}
To learn and do three-dimensional electromagnetic simulation using the finite-difference
time-domain (FDTD) method.

Type of material:
\begin{enumerate}
\item Free space
\item Complex dielectric material
\item Frequency-dependent material
\end{enumerate}
\end{frame}

\begin{frame}
\frametitle{}
Some choice that have been made:
\begin{enumerate}
\item The use of Normalised Units
Maxwell's equations have been normalized by substituting
\begin{equation*}
\widetilde{E} = \displaystyle \sqrt{\frac{\varepsilon_{0}}{\mu_{0}}} E
\end{equation*}

this is a system similar to Gaussian units.

The reason for using it here is the simplicity in the formulation. The $E$ and $H$ fields
have the same order of magnitude. This has an advantage in formulating the PML.

\item Maxwell's Equations with the Flux Density
Time-domain Maxwell's equations from which the FDTD formulation is developed

\begin{equation*}
\frac{\partial E}{\partial t} = \frac{1}{\varepsilon_{0}} \curl{H}
\end{equation*}

\begin{equation*}
\frac{\partial H}{\partial t} = - \frac{1}{\mu_{0}} \curl{E}
\end{equation*}

straight forward formulation


\begin{equation*}
\frac{\partial D}{\partial t} = \curl{H}
\end{equation*}

\begin{equation*}
D = \varepsilon_{0} \varepsilon^{*}_{r} E
\end{equation*}

\begin{equation*}
\frac{\partial H}{\partial t} = - \frac{1}{\mu_{0}} \curl{E}
\end{equation*}

formulation using the flux density in this formulation, it is assumed that the materials
being simulated are non-magnetic, that is, $H = (1/\mu_{0}) B$
\end{enumerate}
\end{frame}

\begin{frame}
\frametitle{Pulse propagating in free space in one dimension}
Time-dependent Maxwell's curl equations for free space:

\begin{equation*}
\frac{\partial E}{\partial t} = \frac{1}{\varepsilon_{0}} \curl{H}
\end{equation*}

\begin{equation*}
\frac{\partial H}{\partial t} = - \frac{1}{\mu_{0}} \curl{E}
\end{equation*}

simple one-dimensional case:
\begin{equation*}
\frac{\partial E_{x}}{\partial t} = - \frac{1}{\varepsilon_{0}} \frac{\partial H_{y}}{\partial z}
\end{equation*}

\begin{equation*}
\frac{\partial H_{y}}{\partial t} = - \frac{1}{\mu_{0}} \frac{\partial E_{x}}{\partial z}
\end{equation*}
\end{frame}

\begin{frame}
\frametitle{}
\begin{itemize}
\item The formulation of Equations assume that the $E$ and $H$ fields are interleaved in both
space and time.
\item The new value of $E_{x}$ is calculated from the previous value of $E_{x}$ and the
most resent values of $H_{y}$. This is the fundamental paradigm of the FDTD method.
\end{itemize}

governing equations,

\begin{equation*}
\widetilde{E}_{x}^{n + 1/2} (k) = \widetilde{E}_{x}^{n - 1/2} (k) - \frac{\Delta
t}{\sqrt{\varepsilon_{0} \mu_{0}} \cdot \Delta x} \Bigg[H_{y}^{n} \left(k + \frac{1}{2}\right)
- H_{y}^{n} \left(k - \frac{1}{2}\right)\Bigg]
\end{equation*}

\begin{equation*}
H_{y}^{n + 1} \left(k + \frac{1}{2}\right) = H_{y}^{n} \left(k + \frac{1}{2}\right) -
\frac{\Delta t}{\sqrt{\varepsilon_{0} \mu_{0}} \cdot \Delta x}  \Bigg[\widetilde{E}_{x}^{n + 1/2} (k + 1) 
- \widetilde{E}_{x}^{n + 1/2} (k)\Bigg]
\end{equation*}
\end{frame}

\begin{frame}

Once the cell size $\Delta x$ is choosen, then the time step $\Delta t$ is determined by

\begin{equation*}
\Delta t = \frac{\Delta x}{2 \cdot c_{0}},
\end{equation*}

where $c_{0}$ is the speed of light in free space. Therefore, remembering that
$\varepsilon_{0} \mu_{0} = 1/(c_{0})^{2}$,

\begin{equation*}
\frac{\Delta t}{\sqrt{\varepsilon_{0} \mu_{0}} \cdot \Delta x} = \frac{\Delta x}{2 \cdot
c_{0}} \cdot \frac{1}{\sqrt{\varepsilon_{0} \mu_{0}} \cdot \Delta x} = \frac{1}{2}
\end{equation*}

\begin{equation*}
\widetilde{E}_{x}^{n + 1/2} (k) = \widetilde{E}_{x}^{n - 1/2} (k) - \frac{1}{2} \Bigg[H_{y}^{n}
\left(k + \frac{1}{2}\right) - H_{y}^{n} \left(k - \frac{1}{2}\right)\Bigg]
\end{equation*}

\begin{equation*}
H_{y}^{n + 1} \left(k + \frac{1}{2}\right) = H_{y}^{n} \left(k + \frac{1}{2}\right) -
\frac{1}{2} \Bigg[\widetilde{E}_{x}^{n + 1/2} (k + 1) - \widetilde{E}_{x}^{n + 1/2}
(k)\Bigg]
\end{equation*}
\end{frame}

\begin{frame}
\frametitle{Simulation in free space}
\begin{figure}[ht!]
\includegraphics[width=1.0\linewidth]{fd1d_1_1.pdf}
\end{figure}
\end{frame}

\begin{frame}
\frametitle{Simulation in free space}
\begin{figure}[ht!]
\includegraphics[width=1.0\linewidth]{fd1d_1_1.pdf}
\end{figure}
\end{frame}

\end{document}
